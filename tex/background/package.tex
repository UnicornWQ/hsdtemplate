% -*- coding=utf-8 -*-
%%%%%%%%%%%%%%%%%%%%%%%%%%%%%%%%%%%%%%%%%%%%%%%%%%%%%%%%%%%%%%%%%%%%%%%%%%%%%%%%%%%%%%%%%%%%%%%
%   河南师范大学研究生毕业论文LaTeX模板
%   使用要求:
%   编译器:XeLatex(2019+),建议关闭拼写检查
%   使用方法:
%   编辑/tex目录下的各个tex文件的文件内容
%   最后编译main.tex
%   警告:需要修改配置就编辑main.tex的内容,编辑内容去/tex目录下的各个文件
%   最后修改日期:2023/10 (王琪)
%	内容结构:
%		文档类型,宏包管理,页面边距,页眉页脚,章节标题,目录设置,参考文献,定理环境,
%		图表环境,代码环境,引用工具,其余设定,正文内容
%%%%%%%%%%%%%%%%%%%%%%%%%%%%%%%%%%%%%%%%%%%%%%%%%%%%%%%%%%%%%%%%%%%%%%%%%%%%%%%%%%%%%%%%%%%%%%%

%=================================文档类型=====================================================
%	毕业论文选取ctexbook比较合适
%	twoside命令,设置为双面排版,左右页边距会根据奇偶页自动调整
%	12pt,字体大小,默认为10pt
%	openright命令,默认openright,即为新的一章在右手边开始

% AutoFakeBold(它会传递给 xeCJK 和 fontspec)来打开全局的伪粗体功能,从而可以使用加粗的宋体。因为TeX本就没有粗体形式的宋体,伪粗体可以模仿得很像。

%=================================宏包管理=====================================================
%	和配置有关的宏包在具体的配置区引用,这里只引用正文区用到的宏包
\usepackage{wallpaper}	% 封面背景包
\usepackage{amsmath,mathtools,amsthm,amsfonts,amssymb,bm}	% AMS包
\usepackage{color}  %字体背景颜色包

%=================================页面边距=====================================================
%	geometry宏包使用教程:http://www.ctex.org/documents/packages/layout/geometry.htm
%	A4纸宽210mm,长297mm
%	left + right + textwidth = 210
%	top + bottom + textheight = 297
%	headheight:页眉文字高度,应当小于等于top
\usepackage{geometry}		% 页面边距包
\geometry{%
	a4paper,
	left=30mm,
	right=25mm,
	top=28mm,
	bottom=25mm,
	%textheight=244mm,
	%textwidth=155mm,
	headheight=21.7mm
}

%=================================页眉页脚=====================================================
%	fancy宏包使用教程:http://www.ctex.org/documents/packages/layout/fancyhdr.htm
%	fancypagestyle{样式名}可以自定义样式,并通过\pagestyle{样式名}和\thispagestyle{样式名}来使用
%   \leftmark可以获取不带星号的chapter标题内容,\rightmark可以获取到不带星号的section标题内容
%   L, C, R分别表示左中右,
%   E, O分别表示偶数页和奇数页
\usepackage{fancyhdr}								% 页眉页脚包
\usepackage{fontspec}
\setmainfont{Times New Roman} % 设置英文字体
\usepackage{setspace} % 段落行距包
\setstretch{1.5} % 设置行距为1.5倍行距
\raggedbottom



%=================================章节标题=====================================================
\usepackage{ctex}
\ctexset{
	chapter = {%
		name = {第,章},
		number = \chinese{chapter},              % 用阿拉伯数字显示章节号
		format += {\bfseries\zihao{3} \centering},       % 设置章节标题为3号黑体且居中
		beforeskip = 36pt,		  % 设置章节标题前的垂直间距为36pt,默认为50pt
		afterskip = 24pt,	   % 设置章节标题后的垂直间距为24pt,默认为40pt
		fixskip = true		 % 设置固定间距为true,抑制标题前后的多余间距
	},
	section = {%
		format = {\songti\zihao{-3}\raggedright},   % section格式添加一条:左对齐
        beforeskip = 18pt,
        afterskip = 18pt
	},
	subsection = {%
		format = {\songti\zihao{4}\raggedright},    % subsection格式添加一条:左对齐
        beforeskip = 12pt,
        afterskip = 12pt
	}
}

%=================================目录设置=====================================================
%	titletoc宏包使用教程:https://blog.csdn.net/golden1314521/article/details/39926135
%						 https://blog.csdn.net/l_changyun/article/details/87431805
%
\usepackage{titletoc}		                % 目录定制包
\renewcommand{\contentsname}{目\quad 录}    % 通过重新定义目录页的标题使得目录中间加上了空格
\titlecontents%	章
	{chapter}[4em]
	{\bfseries\songti\vspace*{7pt}}
	{\contentslabel{4em}}
	{\hspace*{-4em}}
	{~\titlerule*[0.6pc]{$.$}~\contentspage}
\titlecontents%	节
	{section}[4em]
	{\songti}
	{\contentslabel{2em}}
	{\hspace*{-2em}}
	{~\titlerule*[0.6pc]{$.$}~\contentspage}
\titlecontents%	小节
	{subsection}[7em]
	{\songti}
	{\contentslabel{3em}}
	{\hspace*{-2em}}
	{~\titlerule*[0.6pc]{$.$}~\contentspage}

%=================================参考文献=====================================================
%   biblatex宏包使用教程:
%	https://www.overleaf.com/learn/latex/Bibliography_management_with_biblatex
%
\usepackage[%
	backend=biber,				% 设置使用biber进行编译,也可以使用bibtex,但是功能更少
	style=gb7714-2015,			% 设置风格样式为国家标准gb7714-2015
	sorting=none					% 设置排序按照年份,名字,标题进行排序,若想按照引用顺序排序,将其设置为none即可
	]{biblatex}       			% 参考文献包
\addbibresource{bib/ref.bib}    % 加载参考文献的文件

%=================================定理环境=====================================================
% 	自定义定理类环境(定义,引理,定理,推论,例,注)
% 		定理环境命令:\newtheorem{name}[counter]{text}[section]
% 			name:		标识这个环境的关键字(用于编程)
%			counter:	(可选)编号计数器,默认使用自己的计数器,可以传入其他环境的name来共享计数器
% 			text:		真正在文档中打印出来的定理环境的名字
% 			section:	(可选)定理编号依赖的某个章节层次,默认不依赖。
%
\newtheorem{theorem}{\hskip 2em{定理}}[section]
\newtheorem{definition}[theorem]{\hskip 2em{定义}}
\newtheorem{lemma}[theorem]{\hskip 2em{引理}}
\newtheorem{corollary}[theorem]{\hskip 2em{推论}}
% 例,注各自独立编号,无需考虑编号共享的问题,直接创建。证明关键词加粗
\newtheorem{example}{\hskip 2em{例}}[chapter]
\newtheorem{remark}{\hskip 2em{注}}[chapter]
\renewcommand{\proofname}{\hskip 2em \bf 证明}

% 去掉定理后面的小点,不建议使用(默认注释掉)
% \usepackage{xpatch}
% \makeatletter
% \AtBeginDocument{\xpatchcmd{\@thm}{\thm@headpunct{.}}{\thm@headpunct{}}{}{}}
% \makeatother

% 公式按section编号,若想按chapter编号,注释掉这条即可
%\numberwithin{equation}{section}

%=================================图表环境=====================================================
% enumitem宏包设置参考自中南大学学位论文模板
\usepackage[inline]{enumitem}					% 列表工具包
\usepackage{graphicx,ragged2e}							% 插图工具包
\usepackage{subcaption}							% 子图标题包
\usepackage{bicaption}							% 图片标题包
\setlist{%	设置列表样式
	topsep=0.3em, 			% 列表顶端的垂直空白
	partopsep=0pt, 			% 列表环境前面紧接着一个空白行时其顶端的额外垂直空白
	itemsep=0ex plus 0.1ex, % 列表项之间的额外垂直空白
	parsep=0pt, 			% 列表项内的段落之间的垂直空白
	leftmargin=1.5em, 		% 环境的左边界和列表之间的水平距离
	rightmargin=0em, 		% 环境的右边界和列表之间的水平距离
	labelsep=0.5em, 		% 包含标签的盒子与列表项的第一行文本之间的间隔
	labelwidth=2em 			% 包含标签的盒子的正常宽度;若实际宽度更宽,则使用实际宽度。
}
\graphicspath{figure/}		% 设置图片存放目录
\usepackage{longtable,booktabs} % 跨页长表格

%=================================代码环境=====================================================
% 使用listings宏包来插入代码
\usepackage{listings}	% 代码环境包
\renewcommand{\lstlistingname}{算法}	% 重命名代码块标题为算法,例如:算法1.2
\lstset{% 设置算法样式
	keywordstyle=\bfseries, % 设置关键词加粗
	basicstyle=\ttfamily, 	% 设置基础样式字体为等宽
	commentstyle=\ttfamily, % 基本和注释的字体都使用默认的等宽,而非texlive调用的中文字体
	showstringspaces=false, % 不显示中间的空格
	breaklines=true,  		% 对过长的代码自动换行
	frame=single  			% 边框
}

%=================================引用工具=====================================================
% hyperref宏包教程https://www.jianshu.com/p/58e7d0a6d97a
% 实现超链接功能
\usepackage{hyperref}	% 交叉引用包
\hypersetup{%	设置交叉引用属性
	colorlinks=true,	% 设置可跳转的链接为颜色,而不是方框
	urlcolor=black,		% 设置各种链接的颜色均为黑色
	linkcolor=black,
	anchorcolor=black,
	citecolor=black
}

%=================================其余设定=====================================================
% 重新定义一些常用的数学符号
\renewcommand{\Re}{\operatorname{Re}}
\renewcommand{\Im}{\operatorname{Im}}
\newcommand{\mi}{\mathrm{i}}
\newcommand{\md}{\mathrm{d}}
\newcommand{\me}{\mathrm{e}}
\usepackage{indentfirst} %段落首行缩进命令包
\setlength{\parindent}{2em}

